%%%%%%%%%%%%%%%%%%%%%%%%%%%%%%%%%%%%%%%%%
% Short Sectioned Assignment
% LaTeX Template
% Version 1.0 (5/5/12)
%
% This template has been downloaded from:
% http://www.LaTeXTemplates.com
%
% Original author:
% Frits Wenneker (http://www.howtotex.com)
%
% License:
% CC BY-NC-SA 3.0 (http://creativecommons.org/licenses/by-nc-sa/3.0/)
%
%%%%%%%%%%%%%%%%%%%%%%%%%%%%%%%%%%%%%%%%%

%----------------------------------------------------------------------------------------
%	PACKAGES AND OTHER DOCUMENT CONFIGURATIONS
%----------------------------------------------------------------------------------------

\documentclass[paper=a4, fontsize=11pt]{scrartcl} % A4 paper and 11pt font size

\usepackage[T1]{fontenc} % Use 8-bit encoding that has 256 glyphs
%\usepackage{fourier} % Use the Adobe Utopia font for the document - comment this line to return to the LaTeX default
\usepackage[english]{babel} % English language/hyphenation
\usepackage{amsmath,amsfonts,amsthm} % Math packages
\usepackage{bm}
\usepackage{lipsum} % Used for inserting dummy 'Lorem ipsum' text into the template
\usepackage{graphicx} % This one is for pictures
\usepackage{sectsty} % Allows customizing section commands
\allsectionsfont{\centering \normalfont\scshape} % Make all sections centered, the default font and small caps
\usepackage{color}
\usepackage{float}
\floatplacement{figure}{H}
\usepackage{fancyhdr} % Custom headers and footers
\pagestyle{fancyplain} % Makes all pages in the document conform to the custom headers and footers
\fancyhead{} % No page header - if you want one, create it in the same way as the footers below
\fancyfoot[L]{} % Empty left footer
\fancyfoot[C]{} % Empty center footer
\fancyfoot[R]{\thepage} % Page numbering for right footer
\renewcommand{\headrulewidth}{0pt} % Remove header underlines
\renewcommand{\footrulewidth}{0pt} % Remove footer underlines
\setlength{\headheight}{13.6pt} % Customize the height of the header

%\numberwithin{equation}{section} % Number equations within sections (i.e. 1.1, 1.2, 2.1, 2.2 instead of 1, 2, 3, 4)
%\numberwithin{figure}{section} % Number figures within sections (i.e. 1.1, 1.2, 2.1, 2.2 instead of 1, 2, 3, 4)
%\numberwithin{table}{section} % Number tables within sections (i.e. 1.1, 1.2, 2.1, 2.2 instead of 1, 2, 3, 4)

%\setlength\parindent{0pt} % Removes all indentation from paragraphs - comment this line for an assignment with lots of text

%----------------------------------------------------------------------------------------
%	TITLE SECTION
%----------------------------------------------------------------------------------------

\newcommand{\horrule}[1]{\rule{\linewidth}{#1}} % Create horizontal rule command with 1 argument of height

\title{	
\normalfont \normalsize 
\textsc{Columbia University -- Fall 2013} \\ [25pt] % Your university, school and/or department name(s)
\horrule{0.5pt} \\[0.4cm] % Thin top horizontal rule
\huge Machine Learning Homework \#5\\ % The assignment title
\horrule{2pt} \\[0.5cm] % Thick bottom horizontal rule
}

\author{Joe Ellis - jge2105} % Your name

\date{\normalsize\today} % Today's date or a custom date

\begin{document}

\maketitle % Print the title

%----------------------------------------------------------------------------------------
%	PROBLEM 1
%----------------------------------------------------------------------------------------

\section{Problem 1}
Assume you are a contestant of a game show in which you are presented with three closed doors A, B, and C. 
Behind one of the doors is a car which will be yours if you choose the right door. 
After you have randomly (as you have no prior information) selected a door (say door A), the game host opens door B which has nothing inside, while keeping door A and C closed. 
The host then asks whether you want to change your selection from A to C. Should you change?

We should change.
Assuming that the game is fair and the likelihood that the car is behind doors A,B,and C is equal, then $p(car = A) = p(car = B) = p(car = C) = \frac{1}{3}$.
Therefore, without loss of generality if we choose door A, we have a $\frac{1}{3}$ chance of winning the car.
However, the host then opens up one of the doors that is not the door that was chosen or the one that actually contains the car, WLOG let's assume it is door B.

The host then asks us if we would like to change our choice.  
Now there are two doors available to choose from (A and C), and if we were to choose randomly from this configuration we would have $p(car = A) = p(car = C) = \frac{1}{2}$.
However, because the host is guaranteed not to choose the door that we first chose whether it had the car in it or not, the $p(car = chosen (A)) = \frac{1}{3}$ and does not change.
Therefore, we should switch our pick to the other available door, because by switching we upgrade our probability of getting the car to $\frac{1}{2}$.

%----------------------------------------------------------------------------------------
%	PROBLEM 2
%----------------------------------------------------------------------------------------

\section{Problem 2}
\subsection{Problem 2a}
We want to prove that $X \perp Y|Z$ if and only if the joint probability $p(x,y,z)$ can be expressed in the form of $a(x,z)b(y,z)$.
\paragraph{PROOF}
Assume that $X \perp Y|Z$, therefore we know that $p(x|y,z) = p(x,z)$ and $p(x|y) \neq p(x)$.

\begin{align}
p(x,y,z) &= p(y,z)p(x|y,z) \\
&= p(y,z)p(x,z).
\end{align}

Therefore, we have $p(x,y,z)$ can be expressed in the form of $a(x,z)b(y,z)$.

\subsection{Problem 2b}
 We want to prove or disprove with a counter example, that $X \perp Y|Z$ and $X \perp W|(Y,Z)$ implies that $X \perp (W,Y)|Z$.
We want to use this knowledge to prove that $p(x|(w,y,z)) = p(x|z)$.

\paragraph{PROOF}
\begin{align}
p(x|(w,y,z)) &= p(x|y,z),\ by \ X \perp W|(Y,Z) \\
&= p(x|z), \ by \  X \perp Y|Z.
\end{align}

Thus, we have shown $X \perp (W,Y)|Z$.

\subsection(Problem 2c}

\subsection{Bonus}

%----------------------------------------------------------------------------------------
%	PROBLEM 3
%----------------------------------------------------------------------------------------

\section{Problem 3}

%----------------------------------------------------------------------------------------
%	PROBLEM 4
%----------------------------------------------------------------------------------------

 \section{Problem 4}






\end{document}